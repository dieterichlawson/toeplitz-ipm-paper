\documentclass{article}

\usepackage{graphicx} % Required to insert images
\usepackage{listings} % Required for insertion of code
\usepackage{courier} % Required for the courier font
\usepackage{todonotes}
\usepackage{amsmath}
\usepackage{amssymb}
\usepackage{epstopdf}
\usepackage{enumerate}
\usepackage[margin=1.0in]{geometry}

% homework section framework, etc...
\input defs.tex
\title{Implementation of interior-point methods for positive semi-definite Toeplitz cone programming}
\author{Matt Staib, Dieterich Lawson}
\begin{document}
\maketitle
\large
\section{Introduction}
Consider the cone of positive-semidefinite Toeplitz matrices
\[
  \toepcn_n = \{x \in \reals^n \mid \toep(x) \geq 0\}
\]
with $\toep(x)$ defined as

\[
  \toep(x)_{ij} = \begin{cases}
      x_{|i-j|+1} &  0 \leq |i-j| \leq n-1 \\
      0 & \mbox{otherwise}
\end{cases}.
\]

In words, $\toep (x)$ is the square Toeplitz matrix resulting from taking the $i$th
entry of $x$ as the $i$th sub- and super-diagonal. This cone appears in various applications
in signal processing \todo[inline]{Flesh out applications}. In this paper we investigate methods for cone programming
over $\toepcn$, i.e. we present methods for solving problems of the form

\begin{equation}\label{primal}
  \begin{array}{ll}
    \mbox{minimize}    & c^Tx \\
    \mbox{subject to } & Ax = b \\
                       & x \in \toepcn_n. 
  \end{array}
\end{equation}

where $x\in \reals^n$ is the optimization variable, $c \in \reals^n$, $A \in \reals^{m\times n}$, 
and $b \in \reals^m$. The central difficulty for problems of this form is that $\toepcn_n$
is not symmetric, so much of the standard primal-dual machinery becomes unweildly.

The remainder of this paper is organized as follows. We first explore the properties of 
$\toepcn_n$, deriving its dual and showing that it is not symmetric. To motivate
new algorithms we then investigate how un-symmetric cones present a problem for a standard 
primal-dual interior point method. We discuss a suitable barrier for $\toepcn_n$ 
and its use in two different algorithms that make speed gains over the standard primal-dual
interior-point method by forgoing pieces of the dual picture. Finally we present some 
numeric results from these algorithms. Note that nothing presented here is original, and in
many cases follows our sources literally.

\section{The Positive Semi-Definite Toeplitz Cone}

To characterize the dual of $\toepcn_n$ we show that $\toepcn_n$ is the dual of the 
cone of finite autocorrelation sequences. A finite autocorrelation sequence is a vector
$x \in \reals^n$ with a $y \in \reals^n$ such that 
\[
  x_k = \sum_{i=1}^{n-k} y_iy_{i+k} \quad k=1,\ldots,n.
\] 
For any positive scalar $\alpha$, $\alpha x$ can be represented as above 
by $\alpha^{1/2}y$, so the set of finite autocorrelation sequences forms a cone.
The dual of $\corrcn_n$ is 
\[
  \corrcn^*_n = \{z \in \reals^n \mid z^Tx \geq 0 \text{ for } x \in \corrcn_n\}.
\]
If $z^Tx \geq 0$ for all $x \in \corrcn_n$, then for all
$y \in \reals^n$ 
\[
  z^Tx \geq 0 \iff \sum_{i=1}^n z_i \left(\sum_{i=1}^{n-k} y_iy_{i+k}\right).
\]
\todo[inline]{probably don't want that iff symbol in there}
The summation above can be expressed in terms of the Toeplitz form of $z$ ---
the above is true if and only if $y^TT_n(z)y \geq 0$ for all $y \in \reals^n$. Thus
$\corrcn^*_n$ is exactly the positive semi-definite Toeplitz cone, $\toepcn_n$. Nonempty,
closed, convex cones are reflexive, so $\toepcn_n^* = \corrcn_n$, and $\toepcn_n$ is not
symmetric.

\todo[inline]{Do we need more than this? Is it clear that these aren't the same thing or should we provide an example?}
\todo[inline]{Do we need to show that it is nonempty, closed, and convex? Or is that too obvious?}

\section{The Barrier Method}

We turn now to interior point methods and their suitability for
solving un-symmetric cone problems. Consider the following problem, which is slightly 
more general than (\ref{primal})

\begin{equation}\label{primalgen}
  \begin{array}{ll}
    \mbox{minimize}    & c^Tx \\
    \mbox{subject to } & Ax = b \\
                       & x \in K
  \end{array}
\end{equation}
where all variables are as in (\ref{primal}), and $K$ is a proper (closed, convex, solid, pointed) cone 
possessing a self-concordant and $\nu$-logarithmically homogenous barrier function, $\phi$.
A $\nu$ logarithmically homogenous barrier satisfies
\[
  \phi(tx) = -\nu \log(t) + \phi(x).
\]
The \emph{barrier method} is an interior-point method that solves (\ref{primalgen}) by using
$\phi$ to make the cone constraint implicit in the objective. Using equality-constrained
Newton's method, the barrier method solves a series of problems of the form

\begin{equation}\label{primalt}
  \begin{array}{ll}
    \mbox{minimize}    & tc^Tx + \phi(x)\\
    \mbox{subject to } & Ax = b
  \end{array}
\end{equation}
where $t \in \reals,\;t >0$. For each $t >0$, (\ref{primalt}) has a unique minimizer TODO:ref
and that series of minimizers define the \emph{primal central path}
\[
  x^*(t) = \arg\min_x\{tc^Tx + \phi(x) \mid Ax=b\}.
\]
Each point on the primal central path is associated with a dual-feasible point which
can be used to lower-bound the optimal value of (\ref{primalgen}). To see this note that
any minimizer of (\ref{primalt}) $x^*(t)$ must satisfy the optimality conditions
\[
  tc + \nabla\phi(x^*(t)) - y^TA = 0
\]
for some $y\in\reals^m$. It can be shown TODO:cite that if  $x \in \mathbf{int} K$ then
$\nabla\phi(x) \in K^*$. Thus, from the optimality conditions we 
see that $\nabla\phi(x^*(t))/t$, $y/t$ are feasible for the dual of (\ref{primal}). 
The dual is:
\begin{equation}\label{dual}
  \begin{array}{ll}
    \mbox{maximize}    & -b^T\lambda \\
    \mbox{subject to } & c + A^T\lambda + \nu = 0 \\
                       & \nu \in K^* \\
  \end{array}
\end{equation}

where $\lambda, \nu \in \reals^n$ and $K^*$ is the dual cone of $K$. Now we can plug 
$y/t$ into the dual objective, giving:
\[
  c^Tx + b^T(y/t) =   \text{MAGIC} = \nu/t
\]
Which bounds the sub-optimality of 
\[
  \text{something} \leq \nu/t 
\]
This suggests the following algorithm:

\todo[inline]{Algo goes here}

In fact in TODO:Cite Vaderberghe's paper just such a barrier is proposed for the cone we're
interested in. He does lots of stuff to blah \todo[inline]{Section on vanderberghe paper}

\todo[inline]{paragraph on: we would like to do primal-dual IPMs but can't (just give KKT)}
\section{A Predictor-Corrector Method}
Nesterov's method goes here.
\section{Numerical Results}

\end{document}
